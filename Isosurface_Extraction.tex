\section{Isosurface Extraction}
\label{isosurface extraction}

The main aspect of this thesis is the extraction of isosurfaces. The process of converting the voxel grid into a mesh is named \emph{Isosurface extraction} \cite{Len10}. An isoconcentration surface in APT-data context describes a two dimensional space with the property that one atom species is present with a certain concentration (the so called isoconcentration-value). Also other isosurfaces like a surface with equal density or equal stoichiometry are common reconstructions of interest.\\

The way this step is performed has great influence on the resulting visualization. There are a lot of different methods that can be used in order to extract an isosurface from a discrete voxel grid, but with different resulting isosurfaces. Basically the methods are classified into Primal and Dual Methods. The difference between them is the way they locate the topology describing vertices \cite{Lys12}.\\

\subsection{Primal Methods}

The approach of the primal methods is the following\cite{Lys12}:

\begin{enumerate}
    \item Edges crossing the boundary become vertices in the isosurface mesh.
    \item Faces crossing the boundary become edges in the isosurface mesh.
    \item \dots
    \item Cells crossing the boundary become (n-1)-cells in the isosurface mesh.
\end{enumerate}

The most frequently used algorithm for this task is the Marching Cubes \cite{Lor87}\cite{Joh04}, which is a primal method. The basic working principle is to walk iteratively through each cube of the grid and check the vertice signs in each cube. The vertex state is searched in a lookup-table and the related isosurface face is assigned to the current cube. The major drawback of \ac{MC} is a ambiguity arising from different possibilities to connect the calculated vertices.
Based on the \ac{MC} algorithm lots of extensions were developed. An example grounding on \ac{MC} is the \ac{CMS} by \cite{Ho05}.\\

\subsection{Dual Methods}
The second approach are Dual Methods.
The dual methods follow this basic scheme \cite{Lys12}:

\begin{enumerate}
    \item For every edge crossing the boundary, create an (n-1) cell.  (Face in 3D)
    \item For every face crossing the boundary, create an (n-2) cell. (Edge in 3D)
    \item \dots
    \item For every d-dimensional cell, create an (n-d) cell.
    \item \dots
    \item For every n-cell, create a vertex.
\end{enumerate}

A famous representative of dual methods is the SurfaceNets algorithm by \cite{Gib98}. Here the first step is to check the grid for edges that contain a sign change. In three dimensions every edge belongs to four cubes. So the connection of the four cubes' representing vertices results in a quad.\\

The relationship between \ac{MC} and SurfaceNets is that they share the primal/dual property.
I.e. the vertices of the SurfaceNets mesh are associated with the faces of the \ac{MC}'s mesh.
The same is valid the other way around. \cite{Ju02}

\subsection{Hybrids of Primal/Dual Methods}

There is another alternative that takes the advantages of both a cube-based method and a dual method. An example for this approach is the \ac{EMC} \cite{Ram08}. The special thing about this method is to take the 'normals associated with the intersection points on the edges of a cube' \cite{Ju02} into consideration. This proceeding allows to detect sharp features inside the current cube. Features are described by a defined cone. Featureless cubes are treated with standard \ac{MC}, the vertices of cubes containing features are generated by minimizing the quadratic function: 
\begin{equation}
	\label{eq:QEF}
	E[x] = \sum\limits_{i} (n_j \cdot (x-p_i))^2
\end{equation} 

Based on \ac{EMC} and SurfaceNets algorithm Ju et al. \cite{Ju02} proposed a new method called '\emph{Dual Contouring}'.
It runs basically two steps for vertex generation and connecting them to surface simplices.

\begin{enumerate}
	\item For each cube that exhibits a sign change, generate a vertex positioned at the minimizer of 		\cref{eq:QEF} (cf. \ac{EMC}).
	\item For each edge that exhibits a sign change, generate a quad connecting the minimizing 				vertices of the four cubes containing the edge (cf. SurfaceNets).
\end{enumerate}

This proceeding has the great advantage that there is no need to specifically test for features. This algorithm requires the presence of Hermite Data, i.e. for every edge in the voxel grid containing a sign change the intersection point of the contour is known. In addition to this the gradient (the first partial derivations) of the contour has to be calculated. As the gradient is perpendicular to the contour it can be interpreted as the normals of the isosurface.\\

This is schematically shown in \cref{fig:signed_grid}.

\begin{figure}[H]
	\centering
	\includegraphics[width=0.45\textwidth]{signed_grid}
	\caption{Hermite data (Intersection points and its gradients) on a signed grid. \label{fig:signed_grid}}
	\source{\cite{Ju02}}
\end{figure} 

The resulting positioning of vertices with \ac{DC} based on the Hermite Data from above is shown in figure \cref{fig:dc_grid}

\begin{figure}[H]
	\centering
	\includegraphics[width=0.45\textwidth]{dc_grid}
	\caption{Vertex generation with the dual contouring method. \label{fig:dc_grid}}
	\source{\cite{Ju02}}
\end{figure} 

There are several other approaches on this problem. Without detailed explanation these are listed below.
\begin{itemize}
	\item \ac{DMC} \cite{Sch04}
	\item \ac{ASC} \cite{Pos98}
\end{itemize}


