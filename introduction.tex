\chapter{Introduction}
%\chapter{Einleitung}
\label{sec:introduction}

%general motivation for your work, context and goals: 1-2 pages
%
%\begin{itemize}
%\item \textbf{Context:} make sure to link where your work fits in
%\item \textbf{Problem:} gap in knowledge, too expensive, too slow, a deficiency, superseded technology
%\item \textbf{Strategy:} the way you will address the problem
%\end{itemize}

The characterization of nano-scaled materials becomes more and more important these days. This is due to the increasing interest of research in these materials and the downscaling of components in various fields of industries. The \ac{APT} is an analysis technique that is able to resolve a material's structure on the atomic scale. 
It combines the methods \ac{FIM} and \ac{TOFMS}. A sharp tip, connected as positive electrode, becomes layerwisely stripped by the process of field evaporation. This effect is achieved by applying a high eletric field to the material causing \emph{quantum tunneling} of the ionized atoms on the surface. A negatively biased counter electrode accelerates the detached ions on a two dimensional position detector. The species of each atom can be determined by measuring its time of flight. Thus it is possible to assign the detected positions to both a layer of the material and its element species. The third dimension can be reconstructed by subsequently compose the detected layers \cite{Hel02}. \\

The spatial reconstruction of the measured data can be used to visualize the result in three dimensions in order to get further understanding of the experimental outcome.