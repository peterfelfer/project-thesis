\chapter{Conclusion}
\label{conlclusion}

In the previous sections some concepts were described, which all are tools for developing a consistent analysis tool.\\
That is the process of a memory efficient voxelization that simplifies working and processing with the data.
Secondly there is the main task, the extraction of isosurfaces. This step requires the most attention as it is the core problem to solve and has the greatest potential of improvement.
At last there are some analytical methods in order to gain quantitative information from the experiment. These are the calculation of proximity histograms and out of that the determination of the surface excess of solute.